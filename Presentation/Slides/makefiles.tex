\section{Make and Makefiles}

\begin{frame}{What is Make?}
    By definition,
    \begin{center}
        \itshape
        \alert{GNU Make} is a tool which controls the generation of executables and other non-source files of a program from the program's source files.
    \end{center}

    In this regard, make is a \emph{build system}, a tool that automates the action of \textit{building} an application, i.e. it collects dependencies, creates objects, and it links them. 

    Make is a programming language that enables to abstract the build procedure, providing simpler way to compile executables using variables and rules.
\end{frame}

\begin{frame}{Why Make?}
    Make addresses all the limitations of the manual/scripted compilation process:
    \begin{itemize}
        \item \emph{reproducibility}: the build process is defined in a single file, so it can be easily shared and executed on different machines;
        \item \emph{efficiency}: make \textbf{only recompiles} the files that have changed;
        \item \emph{maintainability}: using variables and rules, the build process is more readable and maintainable.
    \end{itemize}
\end{frame}

\begin{frame}{How to use Make?}
    To use make to build a project, it is necessary to create a \emph{Makefile} containing all the rules to build the project.

    After that, when inside the project directory, we can run the command \texttt{make} to build the project.
\end{frame}

\begin{frame}[fragile]{Makefiles}
    Let us consider the compile script we have seen before and the corresponding \textit{Makefile}:
    \begin{columns}[c]
      \begin{column}{0.48\textwidth}
        \begin{lstlisting}
gcc -c lib.c 
gcc -c main.c 
gcc lib.o main.o -o main \end{lstlisting}
      \end{column}
      \begin{column}{0.48\textwidth}
          \begin{lstlisting}
main: lib.o main.o
    gcc lib.o main.o -o main

lib.o: lib.c
    gcc -c lib.c

main.o: main.c 
    gcc -c main.c\end{lstlisting}

      \end{column}
    \end{columns}

\end{frame}

\begin{frame}[fragile]{Makefiles}
    \begin{columns}[c]
      \begin{column}{0.48\textwidth}
          Each Makefile contains a set of \emph{rules} of the form
        \begin{lstlisting}
<target>:<dependencies>
    <steps>\end{lstlisting}
        \begin{itemize}
            \item \emph{target} is the output of the rule;
            \item \emph{dependencies} are the required rules to build the target;
            \item \emph{steps} are all commands required to generate the target.
        \end{itemize}
      \end{column}
      \begin{column}{0.48\textwidth}
          \begin{lstlisting}
main: lib.o main.o
    gcc lib.o main.o -o main

lib.o: lib.c
    gcc -c lib.c

main.o: main.c 
    gcc -c main.c\end{lstlisting}

      \end{column}
    \end{columns}
\end{frame}

\begin{frame}[fragile]{Makefiles: so what?}
    \begin{columns}[c]
      \begin{column}{0.48\textwidth}
          At this stage, the only advantage bring by make is that only source code that will be actually modified is re-compiled. \vspace{3mm}

          To improve, we will start making use of \emph{regular expressions} and \emph{variables}.
      \end{column}
      \begin{column}{0.48\textwidth}
          \begin{lstlisting}
main: lib.o main.o
    gcc lib.o main.o -o main

lib.o: lib.c
    gcc -c lib.c

main.o: main.c 
    gcc -c main.c\end{lstlisting}

      \end{column}
    \end{columns}
\end{frame}

\begin{frame}[fragile]{Makefiles: variables}
    \begin{columns}[c]
      \begin{column}{0.48\textwidth}
          In Makefiles we can define \emph{variables} that can be used in the code as:
        \begin{lstlisting}
<NAME>=<VALUE>\end{lstlisting} \vspace{3mm}
    
        \texttt{\$(<NAME>)} denotes the access of a variable. \vspace{3mm}

        Two \textit{special} variables:
        \begin{itemize}
            \makeatletter
            \item \texttt{\$@}: current target;
            \item \texttt{\$\^}: target dependencies;
            \item \texttt{\$<}: first target dependency;
            \item \texttt{\$?}: \textit{updated} target dependencies;
        \end{itemize}
        

        
      \end{column}
      \begin{column}{0.52\textwidth}
          \begin{lstlisting}
CC=gcc
CFLAGS=-Wall -Wextra
OBJS=lib.o main.o
BINARY=main

$(BINARY): $(OBJS)
    $(CC) -o $@ $^ 

lib.o: lib.c
    $(CC) $(CFLAGS) -c lib.c

main.o: main.c 
    $(CC) $(CFLAGS) -c main.c\end{lstlisting}

      \end{column}
    \end{columns}
\end{frame}

\begin{frame}[fragile]{Makefiles: regular expressions}
    \begin{columns}[c]
      \begin{column}{0.42\textwidth}
          By using the \emph{wildcard} \texttt{\%} (or, in general, any regular expression), we can create a more \textit{general} makefile.
      \end{column}
      \begin{column}{0.54\textwidth}
          \begin{lstlisting}
CC=gcc
CFLAGS=-Wall -Wextra

main: lib.o main.o
    $(CC) -o $@ $^ 

%.o: %.c
    $(CC) $(CFLAGS) -c $^ -o $@\end{lstlisting}

      \end{column}
    \end{columns}
\end{frame}

\begin{frame}[fragile]{Makefiles: targets}
    \begin{columns}[c]
      \begin{column}{0.42\textwidth}
          Within the same makefile, it is possible to create multiple targets.  \vspace{3mm}

          Running the plain \texttt{make} command will build, by default, the top-most target (\texttt{all} in this case). \vspace{3mm}

          But we can also specify the target to generate, like \texttt{make clean}.
      \end{column}
      \begin{column}{0.54\textwidth}
          \begin{lstlisting}
CC=gcc

all: main

clean:
    rm -f *.o main

main: lib.o main.o
    $(CC) -o $@ $^ 

%.o: %.c
    $(CC) -c $^ -o $@\end{lstlisting}

      \end{column}
    \end{columns}
\end{frame}

\begin{frame}[fragile]{Makefiles: suppressing commands}
    When running, make will display the actual shell command that's executed (e.g. \textit{gcc -o lib lib.c ...}).

    To avoid this behavior, it is simply necessary to prepend a \texttt{@} character at the beginning of the line, e.g.:
    \begin{lstlisting}
clean:
    @echo "Removing .o and .s files"
    @rm -f *.o *.s\end{lstlisting}
\end{frame}

\begin{frame}[fragile]{Makefiles: target dependencies}
    In practice, make decides to rebuild a rule when the provided dependencies have \textit{newer} updates w.r.t. to the target itself.

    This, however, doesn't capture all the semantic dependencies between files. \\
    For instance, changing a header \textit{.h, .hpp} won't trigger the compilation of all objects that depends from them (since they are not explicit dependencies).
\end{frame}

\begin{frame}[fragile]{Makefiles: embed header dependencies}
    To embed relationship between source code and header:
    \begin{itemize}
        \item add the \texttt{-MP -MD} compilation flag to generate \textit{.d} dependency files;
        \item append \texttt{-include <files...>.d} at the end of the makefile.
    \end{itemize}
\end{frame}

\begin{frame}{A Makefile example}
    \centering
    \textit{Let's build a makefile for the project in the git repository}
\end{frame}

\begin{frame}{Why not to use Make?}
    Make is a powerful build system, better then basic shell scripting. However:
    \begin{itemize}
        \item as the project gets bigger, it is \textit{harder} to maintain;
        \item it requires knowledge of CLI tools to accomplish multiple tasks;
        \item it is hard to manage external dependencies (e.g. other libraries).
    \end{itemize}

\end{frame}
