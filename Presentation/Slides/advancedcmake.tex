\section{Advanced CMake}
\begin{frame}{Advanced CMake}
    Until now we used the basic features of CMake, showing the potential over plain Make.

From now, we will see some advanced features that are necessary to handle more complex, real-world projects.
\end{frame}

\begin{frame}[fragile]{CMake: finding packages}
    Any decently-complex \emph{project} will \emph{depend on external libraries} (e.g., \textit{Eigen} for linear algebra, \textit{OpenCV} for image processing, etc.).

    To ensure that such dependencies are met, we can use the \href{https://cmake.org/cmake/help/latest/command/find_package.html}{\texttt{find\_package}} command.

    Libraries are searched system-wide, or in directories specified by the \texttt{CMAKE\_PREFIX\_PATH} variable. Suggestions on a per-package basis are avaible.
\end{frame}

\begin{frame}[fragile]{CMake: finding packages}
\begin{lstlisting}
find_package(Eigen3 REQUIRED)
find_package(OpenCV)
\end{lstlisting}
    
    In this example, not finding the \textit{Eigen} library will abort the configuration step, since we set the \texttt{REQUIRED} keyword.

    The \texttt{OpenCV} package is instead optional, i.e., failing to find it will not stop the project build. \\
    The call to \texttt{find\_package} will set some variables, like \texttt{OpenCV\_FOUND} which can be queried to assess wethever the package was found (or not).
\end{frame}

\begin{frame}[fragile]{CMake: using packages}
    Once the package is found, it can be used as any other target in cmake, e.g. by linking it to a target:
\begin{lstlisting}
add_executable(my_target main.cpp)
target_link_libraries(my_target Eigen3::Eigen)
\end{lstlisting}
    This will link the \texttt{my\_target} executable with the \textit{Eigen} library.
    
    In this case, we use a \textit{namespaced} target linking, which is a modern (and recommended) way to handle dependencies in CMake, i.e., of the form \texttt{<package\_name>::<target>}.
\end{frame}

\begin{frame}[fragile]{CMake: fetching content}
    To easen development, cmake provides some useful commands to integrate different projects.

    In this regard, \href{https://cmake.org/cmake/help/latest/module/FetchContent.html}{\texttt{FetchContent}} is a powerful tool to download external projects and build them along with your application.

\begin{lstlisting}
include(FetchContent)
FetchContent_Declare(
    Eigen3
    URL https://gitlab.com/libeigen/eigen/-/archive/3.4.0/eigen-3.4.0.zip
    FIND_PACKAGE_ARGS NAMES Eigen3
)
FetchContent_MakeAvailable(Eigen3)
\end{lstlisting}
\end{frame}

\begin{frame}[fragile]{CMake: file configuration}
    Sometimes it might be necessary to set some variables of the project at configuration time, e.g., the project version or some path.

    To this extent, \href{https://cmake.org/cmake/help/latest/module/FetchContent.html}{\texttt{configure\_file}} is a function that takes as input a template file and write a new file with cmake variable substituted:

\begin{lstlisting}
configure_file(
    defines.hpp.in
    defines.hpp
)
\end{lstlisting}
\end{frame}
