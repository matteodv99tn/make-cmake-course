\section{CMake}
\begin{frame}{What is CMake?}

    \emph{CMake} might be regarded as an abstraction on top of makefiles.

    Indeed, cmake is \emph{not a build system} by itself, but rather a multi-platform \emph{generator for build system}.

    CMake can in fact generate configuration files for make, \textit{ninja} or \textit{visual studio projects}.
    
\end{frame}

\begin{frame}[fragile]{CMake in 3 lines}

    CMake projects are configured in \emph{CMakeLists.txt} files.

    A simple example is
    \begin{lstlisting}
cmake_minimum_required(VERSION 3.11)
project(my_sample_project)

add_executable(hello main.cpp)\end{lstlisting}

    in which we create an executable \textit{hello} starting from a single source file \textit{main.cpp}.

\end{frame}

\begin{frame}[fragile]{CMake in 3 lines: and now?}
    With the \textit{CMakeLists.txt} file in place, we have to generate the file for our build system.

    One cool feature of cmake is that it enables to easily \emph{decouple} the \emph{source code} from the \emph{build} (they can live in different directories in the system).

    A known standard is to create a \textit{build} folder inside the root directory of the project.
\end{frame}

\begin{frame}[fragile]{CMake in 3 lines: and now?}
    With the \textit{CMakeLists.txt} file in place, we have to generate the file for our build system.

    To list availables \emph{generators}, run the command \texttt{cmake -G}. 

    From within the \textit{build} directory, we \emph{configure} the project with the command\\
    \texttt{cmake <path/to/prj> -G makefiles}

    If we don't specify the \texttt{-G} flag, we use the default generator (most probably \textit{make} or \textit{ninja}).

\end{frame}

\begin{frame}[fragile]{CMake in 3 lines: and now?}
    With the \textit{CMakeLists.txt} file in place, we have to generate the file for our build system.

    From within the \textit{build} directory, we configure the project with the command\\
    \texttt{cmake <path/to/prj> -G makefiles}

    We can now \emph{call} the build system to build the project, i.e., running \texttt{make} (or the chosen generator).

\end{frame}

\begin{frame}[fragile]{CMake: generate and build}
    To have access to a full list of arguments, call \texttt{cmake -h}. Generally, to generate a project:
\begin{lstlisting}
cmake [options] <path/to/source>
cmake [options] <path/to/existing/build>
cmake [options] -S <path/to/source> -B <path/to/build>
\end{lstlisting}

Then, to build a project (regardless of the generator), call
\begin{lstlisting}
cmake --build <path/to/build>
\end{lstlisting}

\end{frame}


\begin{frame}[fragile]{CMake: executables and libraries}
    With cmake we can mainly handle 2 kind of \textit{objects}:
    \begin{itemize}
        \item \emph{executables}, so application that \textit{can run} (having a \textit{main} function);
            \begin{lstlisting}               
add_executable(<exec name> <sources...>)
\end{lstlisting}
            
        \item \emph{libraries}, so binary objects that are not executables but contains implementaiton of functions.
            \begin{lstlisting}               
add_library(<lib name> <STATIC | SHARED> <srcs...>)
\end{lstlisting}
    \end{itemize}
\end{frame}

% \begin{frame}[fragile]{CMake: adding sources}
%     Even though it's not a good practise, it is possible to add source files after the executable/library has been declare through the \texttt{target\_sources} function, e.g.
% \begin{lstlisting}
% add_library(mylib src/lib1.cpp src/lib2.cpp)   
% 
% # ... few lines later
% target_source(mylib src/lib3.cpp)
% \end{lstlisting}
% 
% \end{frame}

\begin{frame}[fragile]{CMake: variables}
    As in make, cmake makes extensive use of \emph{variables}. 

    A variable is defined with the \href{https://cmake.org/cmake/help/latest/command/set.html#set}{\texttt{set}} function and can be accessed with the \alert{\texttt{\$\{\}}} notation:
\begin{lstlisting}
set(LIB_SOURCES src/lib1.cpp src/lib2.cpp)
add_library(mylib ${LIB_SOURCES})
\end{lstlisting}
\end{frame}

\begin{frame}[fragile]{CMake: variables}
    CMake provides different variables that simply cmake scripting, like:
    \begin{itemize}
        \item \texttt{CFLAGS}/\texttt{CXXFLAGS}: custom compilation flags for C/C++ executables;
        \item \texttt{CMAKE\_CXX\_STANDARD}: the C++ standard to use;
        \item \texttt{CMAKE\_BUILD\_TYPE}: the build type:
        \begin{itemize}
            \item \texttt{Debug}: no optimization, debug symbols;
            \item \texttt{MinSizeRel}: optimization for size.
            \item \texttt{RelWithDebInfo}: optimization, debug symbols;
            \item \texttt{Release}: full optimization, no debug symbols.
        \end{itemize}
        
        \item \texttt{CMAKE\_SOURCE\_DIR}: the (absolute) path of the source directory;
        \item \texttt{CMAKE\_BINARY\_DIR}: the (absolute) path of the build directory;
        \item \texttt{CMAKE\_CURRENT\_SOURCE\_DIR}: the directory of the current \textit{CMakeLists.txt}.
    \end{itemize}
\end{frame}

\begin{frame}[fragile]{CMake: interacting with the variables}
    The full state of the cmake project is stored in the \texttt{CMakeCache.txt} file in the build directory.

    Variables can be set, while calling \texttt{cmake} command, passing the \texttt{-D} flag, e.g. 
\begin{lstlisting}
cmake .. -DCMAKE_BUILD_TYPE=Debug   
\end{lstlisting}


To interact dynamically with the variables, there exists two executables: \alert{\texttt{ccmake}}, a CLI tool, and \alert{\texttt{cmake-gui}}, a graphical interface.
\end{frame}

\begin{frame}[fragile]{CMake: headers and libraries}
    The main advantage of using cmake is its ability to easily handle executables headers and dependent libraries.

    To specify the location of the headers of a given target, we use the function
\begin{lstlisting}
target_include_directories(<target> <PUBLIC | PRIVATE | INTERFACE> <include paths...>)
\end{lstlisting}

    Similarly, libraries are linked as
\begin{lstlisting}
target_link_libraries(<target> <PUBLIC | PRIVATE | INTERFACE> <include paths...>)
\end{lstlisting}
\end{frame}

\begin{frame}[fragile]{CMake: public, private and interface}
    For both 
    \href{https://cmake.org/cmake/help/latest/command/target_include_directories.html}{\texttt{target\_include\_directories}} 
    and 
    \href{https://cmake.org/cmake/help/latest/command/target_link_libraries.html#target-link-libraries}{\texttt{target\_link\_libraries}}
    , we have 3 specifiers for the property usage of the current and dependent targets:
    \begin{itemize}
        \item \alert{\texttt{PUBLIC}}: the property is propagated to both the current target and its dependents;
        \item \alert{\texttt{PRIVATE}}: the property is only applied to the current target;
        \item \alert{\texttt{INTERFACE}}: the property is only applied to the dependents of the current target.
    \end{itemize}
\end{frame}

\begin{frame}[fragile]{CMake: subdirectories}
    To simplify the management of a project, cmake allows the use of \emph{subdirectories}. 

    One can write a \textit{CMakeLists.txt} file in a subdirectory and include it in the main one with the \texttt{add\_subdirectory} function:
\begin{lstlisting}
add_subdirectory(<path/to/subdir>)
\end{lstlisting}
    
    This enables to have a \emph{modular} project structure, where the knowledge on \textit{how to build a target} is close to the source code.
\end{frame}

\begin{frame}[fragile]{CMake: file globbing}
    The good cmake practice requires that all source files are explicitly listed in the target defition (mainly for clarity).

    Still, there are few cases in which it is useful to use \emph{file globbing}, e.g. to create executables for each source file in a directory.

    To do so, we can use the \texttt{file} function:
\begin{lstlisting}
file(GLOB SOURCES src/*.cpp)
\end{lstlisting}
    Doing so, we create the variable \texttt{SOURCES} that contains all the \texttt{.cpp} files in the \texttt{src} directory.
\end{frame}


\begin{frame}[fragile]{CMake: file globbing}
    We can also do some cmake scripting to create a target for each source file in a directory, e.g.
\begin{lstlisting}
file(GLOB SOURCES src/*.cpp)
foreach(source ${SOURCES})
    get_filename_component(exec_name ${source} NAME_WE)
    add_executable(${exec_name} ${source})
endforeach()
\end{lstlisting}
\metroset{block=fill}
\begin{block}{Note}
    This globbing occurs at \emph{configure} time. 
    If a new file is added to the directory, the project must be reconfigured, otherwise the new file will not be included in the build.
\end{block}
\end{frame}

\begin{frame}[fragile]{CMake: logging}
    As scripts become more complex, it is useful to have a way to log messages to the console. 

    For this purpose, cmake provides the \href{https://cmake.org/cmake/help/latest/command/message.html}{\texttt{message}} function:
\begin{lstlisting}
message(<mode> "<message>")
\end{lstlisting}
where \texttt{mode} could be \texttt{STATUS}, \texttt{WARNING}, \texttt{SEND\_ERROR}, \texttt{FATAL\_ERROR} or \texttt{AUTHOR\_WARNING}.
\end{frame}

\begin{frame}[fragile]{CMake: functions and macros}
    To avoid code duplication, cmake allows the definition of \emph{functions} and \emph{macros}. 

    Their syntax is similar, but the main difference is that functions have their own scope, while macros are expanded in the caller's scope (are \textit{inlined}).

    \begin{columns}[c]
      \begin{column}{0.48\textwidth}
\begin{lstlisting}
function(<name> <args ...>)
    <body>
endfunction()
\end{lstlisting}
      \end{column}
      \begin{column}{0.46\textwidth}
\begin{lstlisting}
macro(<name> <args ...>)
    <body>
endmacro()
\end{lstlisting}
      \end{column}
    \end{columns}
    
\end{frame}

\begin{frame}[fragile]{CMake: a function example}
    An example of function is the one printing the name of the variable, as well as its value:
\begin{lstlisting}
function(print var)    
   message(STATUS " > ${var} = ${${var}}") 
endfunction()
\end{lstlisting} \vspace{3mm}

    \begin{columns}[t]
      \begin{column}{0.48\textwidth}
          The script
\begin{lstlisting}
set(MY_VAR "Hello World")
print(MY_VAR)
print(MY_UNSET_VAR)
\end{lstlisting}
      \end{column}
      \begin{column}{0.46\textwidth}
          Produces the output
          \begin{verbatim}
 > MY_VAR = Hello World
 > MY_UNSET_VAR =\end{verbatim}
      \end{column}
    \end{columns}

\end{frame}

\begin{frame}[fragile]{CMake: function with variable number of arguments}
    Function can accept a variable number of arguments, and provide default variable names to access them, particularly:
    \begin{itemize}
        \item \texttt{ARGC}: the number of arguments;
        \item \texttt{ARGN}: the full list of arguments;
        \item \texttt{ARGV0}, \texttt{ARGV1}, \texttt{ARGV2}, \ldots: the first, second, ... argument.
    \end{itemize}
\end{frame}

\begin{frame}[fragile]{CMake: function with variable number of arguments}
    In light of this, we can define a function that prints all its arguments:
    \begin{lstlisting}
function(print)
    foreach(arg ${ARGN})
        message(STATUS " > ${arg} = ${${arg}}")
    endforeach()
endfunction()
\end{lstlisting}
\end{frame}
