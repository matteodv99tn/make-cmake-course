\section{Compiling the code}
\begin{frame}[fragile]{Hello world!}
    The first thing we can do is the basic \textit{hello world} executable in plain C.

    First let's create a \texttt{hello.c} file:
\begin{lstlisting}[language=C]
#include <stdio.h>

int main() {
    printf("Hello, world!");
    return 0;
} \end{lstlisting}
\end{frame}

\begin{frame}[fragile]{Hello world!}
    We can now switch to the terminal to actually compile the code:

    \begin{lstlisting}
gcc hello.c -o hello \end{lstlisting}

In this command we:
\begin{itemize}
    \item call the \texttt{gcc} compiler;
    \item specify the source file \texttt{hello.c};
    \item specify the output (\texttt{-o}) object \texttt{hello}.
\end{itemize}
\end{frame}

\begin{frame}[fragile]{Hello world!}
    \begin{lstlisting}
gcc hello.c -o hello \end{lstlisting}
    
    Under the hood, this command performs lots of operations:
    \begin{itemize}
        \item it \emph{pre-process} the \textit{.c, .cpp} source code (e.g., header and macro expansion);
        \item it \emph{compiles} the code into assembly code that's specific to the target architecture;
        \item the \emph{assembler} converts the assembly code into machine code (\textit{.o, .obj});
        \item the \emph{linker} links the object files into a final executable (\textit{.out, .exe}).
    \end{itemize}
\end{frame}

\begin{frame}[fragile]{A more complex example}

    Let's create a simple project that involves multiple sources: a library (\textit{lib.c, lib.h}) and a main program \textit{main.c}.

    There are two \textit{equivalent} ways to compile the code: either

    \begin{lstlisting}
gcc lib.c main.c -o main \end{lstlisting}
    or
    \begin{lstlisting}
gcc -c lib.c 
gcc -c main.c 
gcc lib.o main.o -o main \end{lstlisting}
\end{frame}

\begin{frame}[fragile]{Some compilation flags}
    \begin{lstlisting}
gcc -c lib.c \end{lstlisting}

    In this command, the \texttt{-c} flag tells the compiler to compile and assembly the \textit{lib.c} source code, creating \textit{lib.o} (since we didn't specify the output with \texttt{-o}).
\end{frame}

\begin{frame}[fragile]{Some compilation flags}
    A complete list of compiler flags can be found by running
    \begin{lstlisting}
gcc --help \end{lstlisting}

    Some relevant flags are:
    \begin{itemize}
        \item \texttt{-E}: perform only the pre-processing;
        \item \texttt{-S}: perform only the compilation to assembly \textit{.s};
    \end{itemize}

\end{frame}

\begin{frame}[fragile]{Best compilation method}

    What's better between the two methods?
    \begin{columns}[c]
      \begin{column}{0.48\textwidth}
        \begin{lstlisting}
gcc lib.c main.c -o main \end{lstlisting}
      \end{column}
      \begin{column}{0.48\textwidth}
        \begin{lstlisting}
gcc -c lib.c 
gcc -c main.c 
gcc lib.o main.o -o main \end{lstlisting}

      \end{column}
    \end{columns}
\end{frame}

\begin{frame}[fragile]{Simplifying the compilation}

    The work of compilation is tedious and error-prone. Each time we modify our source code, we need to update our objects and link the final executable.

    How can we simplify this process?

    Let's put all the commands into a \emph{shell script}$\,$! \\
    For example, we can create a \texttt{build.sh} file with the following content:
    \begin{lstlisting}
gcc -c lib.c 
gcc -c main.c 
gcc lib.o main.o -o main \end{lstlisting}
\end{frame}

\begin{frame}[fragile]{Shell scripts for compilation}
    Using a shell script is a good idea, but it's not the best solution.

    Indeed there are few drawbacks:
    \begin{itemize}
        \item \textit{.sh} scripts are not scalable;
        \item \textit{.sh} scripts are less configurable;
        \item each time, we need to recompile all the sources. This is no problem for small projects, but it's a big issue for large ones (e.g. the Linux kernel takes hours to compile).
    \end{itemize}
\end{frame}
