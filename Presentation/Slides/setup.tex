\section{Environment setup}
\begin{frame}{The preliminaries}
    Before actually starting the course we need some setup...

    We need:
    \begin{itemize}
        \item a \emph{compiler}; 
        \item the \emph{make} and \emph{cmake} programs;
        \item an \emph{IDE} to modify our code (VSCode).
    \end{itemize}
    
\end{frame}


\begin{frame}[fragile]{Unix systems}
    The \emph{make} program just works on unix-like system, i.e., Linux and MacOs. 
    Windows is not ufficially supported, and thus is not covered in this course, so please use \textit{WSL}, a \textit{docker container} or a \textit{virtual machine}.

    On Ubuntu:
\begin{lstlisting}
sudo apt update
sudo apt install -y build-essential git \end{lstlisting}

    On MacOs, make sure \texttt{xcode} and \texttt{homebrew} are installed, then
\begin{lstlisting}
brew install make\end{lstlisting}

    
    
\end{frame}

\begin{frame}{Compilers}
    Generally speaking, there exists multiple compilers. For unix systems, the most famous ones are the \emph{GNU Compiler Collection} (GCC) and the \emph{LLVM}-based {CLANG}.

    Other known solutions are the Intel C++, and for windows the \emph{Microsoft Visual Studio Compiler} (MVSC).

    We will mainly work with GCC.
\end{frame}


